%!TEX TS-program = xelatex
%!TEX encoding = UTF-8 Unicode
% Awesome CV LaTeX Template for CV/Resume
%
% This template has been downloaded from:
% https://github.com/posquit0/Awesome-CV
%
% Author:
% Claud D. Park <posquit0.bj@gmail.com>
% http://www.posquit0.com
%
%
% Adapted to be an Rmarkdown template by Mitchell O'Hara-Wild
% 23 November 2018
%
% Template license:
% CC BY-SA 4.0 (https://creativecommons.org/licenses/by-sa/4.0/)
%
%-------------------------------------------------------------------------------
% CONFIGURATIONS
%-------------------------------------------------------------------------------
% A4 paper size by default, use 'letterpaper' for US letter
\documentclass[11pt, a4paper]{awesome-cv}

% Configure page margins with geometry
\geometry{left=1.4cm, top=.8cm, right=1.4cm, bottom=1.8cm, footskip=.5cm}

% Specify the location of the included fonts
\fontdir[fonts/]

% Color for highlights
% Awesome Colors: awesome-emerald, awesome-skyblue, awesome-red, awesome-pink, awesome-orange
%                 awesome-nephritis, awesome-concrete, awesome-darknight

\definecolor{awesome}{HTML}{414141}

% Colors for text
% Uncomment if you would like to specify your own color
% \definecolor{darktext}{HTML}{414141}
% \definecolor{text}{HTML}{333333}
% \definecolor{graytext}{HTML}{5D5D5D}
% \definecolor{lighttext}{HTML}{999999}

% Set false if you don't want to highlight section with awesome color
\setbool{acvSectionColorHighlight}{true}

% If you would like to change the social information separator from a pipe (|) to something else
\renewcommand{\acvHeaderSocialSep}{\quad\textbar\quad}

\def\endfirstpage{\newpage}

%-------------------------------------------------------------------------------
%	PERSONAL INFORMATION
%	Comment any of the lines below if they are not required
%-------------------------------------------------------------------------------
% Available options: circle|rectangle,edge/noedge,left/right

\photo{data/avatar.jpg}
\name{Hugo}{Botha}

\position{Assistant Professor}
\address{Department of Neurology, Mayo Clinic}

\email{\href{mailto:botha.hugo@mayo.edu}{\nolinkurl{botha.hugo@mayo.edu}}}
\homepage{www.github.io/HugoBothaMD}
\github{HugoBothaMD}
\twitter{HugoBothaMD}

% \gitlab{gitlab-id}
% \stackoverflow{SO-id}{SO-name}
% \skype{skype-id}
% \reddit{reddit-id}

\quote{I am a behavioral neurologist at Mayo Clinic in Rochester, MN. My
research is focused on understanding the dynamic, system level changes
that characterize degenerative disease, with a focus on speech/language
disorders,Alzheimer's disease, and Alzheimer's mimics. I have a
particular interest in Bayesian methods and unsupervised machine
learning applied to neurological and imaging data.}

\usepackage{booktabs}

\providecommand{\tightlist}{%
	\setlength{\itemsep}{0pt}\setlength{\parskip}{0pt}}

%------------------------------------------------------------------------------


\usepackage{fancyhdr}
\pagestyle{fancy}
\fancyhf{}
\fancyhead[R]{\thepage}

% Pandoc CSL macros
\newlength{\cslhangindent}
\setlength{\cslhangindent}{1.5em}
\newlength{\csllabelwidth}
\setlength{\csllabelwidth}{3em}
\newenvironment{CSLReferences}[3] % #1 hanging-ident, #2 entry spacing
 {% don't indent paragraphs
  \setlength{\parindent}{0pt}
  % turn on hanging indent if param 1 is 1
  \ifodd #1 \everypar{\setlength{\hangindent}{\cslhangindent}}\ignorespaces\fi
  % set entry spacing
  \ifnum #2 > 0
  \setlength{\parskip}{#2\baselineskip}
  \fi
 }%
 {}
\usepackage{calc}
\newcommand{\CSLBlock}[1]{#1\hfill\break}
\newcommand{\CSLLeftMargin}[1]{\parbox[t]{\csllabelwidth}{#1}}
\newcommand{\CSLRightInline}[1]{\parbox[t]{\linewidth - \csllabelwidth}{#1}}
\newcommand{\CSLIndent}[1]{\hspace{\cslhangindent}#1}

\begin{document}

% Print the header with above personal informations
% Give optional argument to change alignment(C: center, L: left, R: right)
\makecvheader

% Print the footer with 3 arguments(<left>, <center>, <right>)
% Leave any of these blank if they are not needed
% 2019-02-14 Chris Umphlett - add flexibility to the document name in footer, rather than have it be static Curriculum Vitae
\makecvfooter
  {February 11, 2021}
    {Hugo Botha~~~·~~~Curriculum Vitae}
  {\thepage}


%-------------------------------------------------------------------------------
%	CV/RESUME CONTENT
%	Each section is imported separately, open each file in turn to modify content
%------------------------------------------------------------------------------



\hypertarget{education}{%
\section{Education}\label{education}}

\begin{cventries}
    \cventry{Clinician Investigator Program - Radiology}{Mayo Clinic}{Rochester, MN, USA}{07/2017-06/2019}{}\vspace{-4.0mm}
    \cventry{Behavioral Neurology Fellowship}{Mayo Clinic}{Rochester, MN, USA}{07/2016-06/2017}{}\vspace{-4.0mm}
    \cventry{Neurology Residency}{Mayo Clinic}{Rochester, MN, USA}{07/2012-06/2016}{}\vspace{-4.0mm}
    \cventry{Bachelor of Medicine Bachelor of Surgery (MBChB)}{University of Stellenbosch}{Cape Town, WC, SA}{01/2006-12/2011}{}\vspace{-4.0mm}
\end{cventries}

\hypertarget{awards-and-honors}{%
\section{Awards and Honors}\label{awards-and-honors}}

\begin{cventries}
    \cventry{Awarded to an individual for outstanding research in the field of behavioral neurology.}{Norman Geschwind Prize Award in Behavioral Neurology}{2021}{American Academy of Neurology}{}\vspace{-4.0mm}
    \cventry{Recognition of outstanding basic or clinical research during residency or fellowship}{Department of Neurology Research Award}{2016}{Mayo Clinic}{}\vspace{-4.0mm}
    \cventry{Awarded for outstanding clinical performance, humanitarianism, and scholarly activity}{Mayo Brother's Fellowship Award}{2016}{Mayo Clinic}{}\vspace{-4.0mm}
    \cventry{Awarded to the best undergraduate student from the graduating classes in health and allied health sciences}{Faculty of Health Sciences Medal for Best Graduating Student}{2011}{University of Stellenbosch}{}\vspace{-4.0mm}
    \cventry{Awarded for outstanding achievement in either academics, sport or culture}{Rector’s Award for Excellent Achievement: Academics}{2011}{University of Stellenbosch}{}\vspace{-4.0mm}
\end{cventries}

\hypertarget{professional-memberships-and-activities}{%
\section{Professional Memberships and
Activities}\label{professional-memberships-and-activities}}

\hypertarget{society-membership}{%
\subsection{Society Membership}\label{society-membership}}

\begin{cventries}
    \cventry{Member}{American Academy of Neurology}{}{2013-present}{}\vspace{-4.0mm}
    \cventry{Member}{International Society to Advance Alzheimer's Research and Treatment}{}{2017-present}{}\vspace{-4.0mm}
    \cventry{Member}{Organization for Human Brain Mapping}{}{2017-present}{}\vspace{-4.0mm}
    \cventry{Member}{International Society for Frontotemporal Dementias}{}{2020-present}{}\vspace{-4.0mm}
    \cventry{Member}{Organization for Computational Neurosciences}{}{2020-present}{}\vspace{-4.0mm}
\end{cventries}

\hypertarget{professional-service}{%
\subsection{Professional Service}\label{professional-service}}

\begin{cventries}
    \cventry{Course Monitor}{Conferences}{}{}{\begin{cvitems}
\item American Academy of Neurology Annual Meeting (2016)
\end{cvitems}}
    \cventry{Abstract Reviewer}{}{}{}{\begin{cvitems}
\item Alzheimer's Association International Conference (2018, 2019, 2020)
\end{cvitems}}
    \cventry{Ad Hoc Reviewer}{Journals}{}{}{\begin{cvitems}
\item Alzheimer's Research \& Therapy
\item Brain
\item Human Brain Mapping
\item JAMA Psychiatry
\item Journal of Alzheimer's Disease
\item Journal of the Neurological Sciences
\item Mayo Clinic Proceedings
\item Medical Image Analysis
\item Nature: Scientific Reports
\item Neurobiology of Aging
\item NeuroImage: Clinical
\item Neurology
\item Neuropsychologia
\item Parkinsonism and Related disorders
\end{cvitems}}
\end{cventries}

\hypertarget{teaching}{%
\section{Teaching}\label{teaching}}

\hypertarget{internal-lectures}{%
\subsection{Internal Lectures}\label{internal-lectures}}

\begin{cventries}
    \cventry{Basic Summer Lecture Series}{Neurology Residency Program}{}{}{\begin{cvitems}
\item Bedside Cognitive Assessment (2018,2019,2020)
\end{cvitems}}
    \cventry{Neuroanatomy Course}{}{}{}{\begin{cvitems}
\item Cortex I:  Cytoarchitecture and Organization (2019, 2020)
\item Cortex II:  Lobes and Functions (2019, 2020)
\end{cvitems}}
\end{cventries}

\hypertarget{internal-conferences}{%
\subsection{Internal Conferences}\label{internal-conferences}}

\begin{cventries}
    \cventry{Grand Rounds}{Neurology Department}{}{}{\begin{cvitems}
\item Neurology Clinicopathologic Conference - A young man with progressive language difficulty (2015)
\end{cvitems}}
    \cventry{Subspecialty Conference}{}{}{}{\begin{cvitems}
\item  Left Speechless: Dealing with the word salad of progressive aphasia and apraxia of speech (2015)
\item The left hand of neurology: sinistrality, hemispheric specialization and neurodegenerative disease (2017)
\end{cvitems}}
\end{cventries}

\hypertarget{lay-public-focused-talks}{%
\subsection{Lay Public Focused Talks}\label{lay-public-focused-talks}}

\begin{cventries}
    \cventry{Posterior Cortical Atrophy Education Seminar}{ADRC Patient-Caregiver Seminars}{}{}{\begin{cvitems}
\item Making a Diagnosis
\end{cvitems}}
    \cventry{CBD and PSP Education Seminar}{}{}{}{\begin{cvitems}
\item  Journey to the Diagnosis (2017)
\end{cvitems}}
    \cventry{Frontotemporal Dementia and Primary Progressive Aphasia Education Seminar}{}{}{}{\begin{cvitems}
\item FTD/PPA: Making a Diagnosis (2017)
\item FTD/PPA: Pathology and Pathophysiology (2017)
\end{cvitems}}
    \cventry{Chesley Center on Aging 2017 Dementia Care Conference}{Invited Talks}{}{}{\begin{cvitems}
\item Latest Findings and Trends in Research (2017)
\end{cvitems}}
\end{cventries}

\hypertarget{institutionaldepartmental-activities}{%
\section{Institutional/Departmental
Activities}\label{institutionaldepartmental-activities}}

\hypertarget{active}{%
\subsection{Active}\label{active}}

\begin{cventries}
    \cventry{Voting Member}{Neurology Research Committee}{2019 - Current}{}{}\vspace{-4.0mm}
    \cventry{Deputy Director}{Neurology Artificial Intelligence Program}{2020 - Current}{}{}\vspace{-4.0mm}
    \cventry{Deputy Director}{Behavioral Neurology Fellowship}{2020 - Current}{}{}\vspace{-4.0mm}
    \cventry{Section Chair (Neurology)}{Machine Learning/Deep Learning Journal Club}{2020 - Current}{}{}\vspace{-4.0mm}
\end{cventries}

\hypertarget{prior}{%
\subsection{Prior}\label{prior}}

\begin{cventries}
    \cventry{Prelim-year representative}{Mayo Internal Medicine Residents Council}{2012 - 2013}{}{}\vspace{-4.0mm}
    \cventry{Voting Member}{Neurology Residency Recruitment Committee}{2014 - 2016}{}{}\vspace{-4.0mm}
\end{cventries}

\hypertarget{mentoring}{%
\section{Mentoring}\label{mentoring}}

\begin{cventries}
    \cventry{Project: Delays in the diagnosis of PPAOS}{Johhny Dang (Medical student)}{2019 - Current}{}{\begin{cvitems}
\item Abstract submitted to AAN 2021
\item 
Manuscript submitted
\end{cvitems}}
    \cventry{Project: Neurologic causes for delusional infestation}{Zeynep Secking (Intern)}{2020 - Current}{}{\begin{cvitems}
\item 
Manuscript in preparation
\end{cvitems}}
    \cventry{Project: FDG-PET to predict decline in atypical Alzheimer's disease}{Ryan Coburn (Neurology Resident)}{2020 - Current}{}{\begin{cvitems}
\item 
Data analysis in progress
\end{cvitems}}
\end{cventries}

\hypertarget{research-funding}{%
\section{Research funding}\label{research-funding}}

\hypertarget{active-1}{%
\subsection{Active}\label{active-1}}

\begin{cventries}
    \cventry{National Institute on Deafness and Other Communication Disorders}{The neurobiology of two distinct types of progressive apraxia of speech (Co-I)}{2017-2022}{R01 DC 14942}{}\vspace{-4.0mm}
    \cventry{National Institute on Deafness and Other Communication Disorders}{Longitudinal multi-modality imaging in progressive apraxia of speech (Co-I)}{2018-2023}{R01 DC 12519}{}\vspace{-4.0mm}
    \cventry{ National Institute on
Aging}{ARTFL LEFFTDS Longitudinal Frontotemporal Lobar Degeneration (ALLFTD) (Co-I)}{2019-2024}{U19 AG 63911}{}\vspace{-4.0mm}
    \cventry{National Institute of
Neurological Disorders and Stroke}{Independent brain metabolic patterns in Progressive Supranuclear Palsy (Co-I)}{2020-2022}{R03 NS 114365}{}\vspace{-4.0mm}
\end{cventries}

\hypertarget{presentations}{%
\section{Presentations}\label{presentations}}

\hypertarget{conference-presentations-10}{%
\subsection{Conference Presentations
(10)}\label{conference-presentations-10}}

\begingroup
\footnotesize
\setlength{\leftskip}{0.1in}
\setlength{\parindent}{-0.1in}

\hypertarget{refs_selected}{}
\leavevmode\hypertarget{ref-Botha_FTD_2014}{}%
\textbf{Botha,H.} (2014). \emph{{Diffusion tensor imaging in primary
progressive aphasia and apraxia of speech}} {[}Poster{]}. {International
Frontotemporal Dementia Conference}, Vancouver, {BC}, {Canada}.

\leavevmode\hypertarget{ref-Botha_ANA_2015}{}%
\textbf{Botha,H.} (2015). \emph{{Category Specific Imaging Correlates in
Neurodegenerative Anomia}} {[}Poster{]}. {American Neurologic
Association Annual Meeting}, Philadelphia, {PA}, {USA}.

\leavevmode\hypertarget{ref-Botha_REST_2016}{}%
\textbf{Botha,H.} (2016). \emph{{Language network dysfunction in primary
progressive apraxia of speech}} {[}Poster{]}. {Fifth Biennial Conference
on Resting-State and Brain Connectivity}, Vienna, {Austria}.

\leavevmode\hypertarget{ref-Botha_KI_2018}{}%
\textbf{Botha,H.} (2018). \emph{{A novel FDG-PET based marker of TDP-43
and hippocampal sclerosis pathology}} {[}Platform{]}. {24th Mayo-KI
Annual Scientific Research Meeting}, Rochester, {MN}, {USA}.

\leavevmode\hypertarget{ref-Botha_HAI_2018}{}%
\textbf{Botha,H.} (2018). \emph{{FDG-PET in tau-negative amnestic
dementia resembles that of autopsy proven hippocampal sclerosis}}
{[}Platform{]}. {Human Amyloid Imaging}, Miami, {FL}, {USA}.

\leavevmode\hypertarget{ref-Botha_AIC_2018}{}%
\textbf{Botha,H.} (2018). \emph{{Variability in pDMN Connectivity and
Relative Hubness in Amyloid-Negative Cognitively Normal Individuals
Predicts Amyloid and Tau Deposition Patterns in Typical Alzheimer's
Disease}} {[}Poster{]}. {Alzheimer's Association International
Conference}, Chicago, {IL}, {USA}.

\leavevmode\hypertarget{ref-Botha_AIC_2019}{}%
\textbf{Botha,H.} (2019). \emph{{Brain States and Tau PET Patterns
Interact across the Aging-Alzheimer's Continuum}} {[}Poster{]}.
{Alzheimer's Association International Conference}, Los Angeles, {CA},
{USA}.

\leavevmode\hypertarget{ref-Botha_HAI_2019}{}%
\textbf{Botha,H.} (2019). \emph{{Data-driven characterization of
cross-sectional and longitudinal molecular imaging in aging and
Alzheimer's disease}} {[}Platform{]}. {Human Amyloid Imaging}, Miami,
{FL}, {USA}.

\leavevmode\hypertarget{ref-Botha_ADRD_2019}{}%
\textbf{Botha,H.} (2019). \emph{{Linking Brain States to Molecular
Pathology across the Aging-Alzheimer's Continuum}} {[}Blitz-poster{]}.
{ADC Annual Conference}, St Louis, {MO}, {USA}.

\leavevmode\hypertarget{ref-Botha_AAIC_2020}{}%
\textbf{Botha,H.} (2020). \emph{{Disrupted brain dynamics across the
Alzheimer's disease spectrum is related to tau accumulation}}
{[}Platform{]}. {Alzheimer's Association International Conference},
Virtual, originally planned for Amsterdam, {NL}.

\endgroup

\hypertarget{bibliography}{%
\section{Bibliography}\label{bibliography}}

\hypertarget{book-chapters-2}{%
\subsection{Book Chapters (2)}\label{book-chapters-2}}

\begingroup
\footnotesize
\setlength{\leftskip}{0.1in}
\setlength{\parindent}{-0.1in}

\hypertarget{refs_bookchp}{}
\leavevmode\hypertarget{ref-botha_functional_2018}{}%
\textbf{Botha,H.}, \& Jones, D. T. (2018). Functional {Connectivity} in
{Dementia}. In C. Habas (Ed.), \emph{The {Neuroimaging} of {Brain}
{Diseases}: {Structural} and {Functional} {Advances}} (pp. 245--266).
Springer International Publishing.
\url{https://doi.org/10.1007/978-3-319-78926-2_11}

\leavevmode\hypertarget{ref-botha_primary_nodate}{}%
\textbf{Botha,H.}, \& Utianski, R. L. (n.d.). Primary {Progressive}
{Apraxia} of {Speech}. In \emph{Primary {Progressive} {Aphasia} and
{Other} {Frontotemporal} {Dementias}: {Diagnosis} and {Treatment} of
{Associated} {Communication} {Disorders}}. Plural Publishing.

\endgroup

\hypertarget{preprints-1}{%
\subsection{Preprints (1)}\label{preprints-1}}

\begingroup
\footnotesize
\setlength{\leftskip}{0.1in}
\setlength{\parindent}{-0.1in}

\hypertarget{refs_preprint}{}
\leavevmode\hypertarget{ref-jones_patterns_2020}{}%
Jones, D., Lowe, V., Graff-Radford, J., \textbf{Botha,H.}, Wiepert, D.,
Murphy, M. C., Murray, M., Senjem, M., Gunter, J., Wiste, H., Boeve, B.,
Knopman, D., Petersen, R., \& Jack, C. (2020). Patterns of
neurodegeneration in dementia reflect a global functional state space.
\emph{medRxiv}. \url{https://doi.org/10.1101/2020.11.09.20228676}

\endgroup

\hypertarget{peer-reviewed-57}{%
\subsection{Peer Reviewed (57)}\label{peer-reviewed-57}}

\begingroup
\footnotesize
\setlength{\leftskip}{0.1in}
\setlength{\parindent}{-0.1in}

\hypertarget{refs_journals}{}
\leavevmode\hypertarget{ref-botha_discourse_2011}{}%
\textbf{Botha,H.}, Schalkwyk, G. van, Bezuidenhout, J., \& Schalkwyk, S.
van. (2011). Discourse of final-year medical students during clinical
case presentations. \emph{African Journal of Health Professions
Education}, \emph{3}(1), 3--6.
\url{https://www.ajol.info/index.php/ajhpe/article/view/69937}

\leavevmode\hypertarget{ref-botha_reliability_2012}{}%
\textbf{Botha,H.}, Ackerman, C., Candy, S., Carr, J. A.,
Griffith-Richards, S., \& Bateman, K. J. (2012). Reliability and
{Diagnostic} {Performance} of {CT} {Imaging} {Criteria} in the
{Diagnosis} of {Tuberculous} {Meningitis}. \emph{PLOS ONE}, \emph{7}(6),
e38982. \url{https://doi.org/10.1371/journal.pone.0038982}

\leavevmode\hypertarget{ref-van_schalkwyk_comparison_2012}{}%
Schalkwyk, G. van, \textbf{Botha,H.}, \& Seedat, S. (2012). Comparison
of 2 {Dementia} {Screeners}, the {Test} {Your} {Memory} {Test} and the
{Mini}-{Mental} {State} {Examination}, in a {Primary} {Care} {Setting}.
\emph{Journal of Geriatric Psychiatry and Neurology}, \emph{25}(2),
85--88. \url{https://doi.org/10.1177/0891988712447895}

\leavevmode\hypertarget{ref-botha_attention_2012}{}%
\textbf{Botha,H.}, \& Carr, J. (2012). Attention and visual dysfunction
in {Parkinson}'s disease. \emph{Parkinsonism \& Related Disorders},
\emph{18}(6), 742--747.
\url{https://doi.org/10.1016/j.parkreldis.2012.03.004}

\leavevmode\hypertarget{ref-botha_pimple_2014}{}%
\textbf{Botha,H.}, Whitwell, J. L., Madhaven, A., Senjem, M. L., Lowe,
V., \& Josephs, K. A. (2014). The pimple sign of progressive
supranuclear palsy syndrome. \emph{Parkinsonism \& Related Disorders},
\emph{20}(2), 180--185.
\url{https://doi.org/10.1016/j.parkreldis.2013.10.023}

\leavevmode\hypertarget{ref-zalewski_fdg-pet_2014}{}%
Zalewski, N., \textbf{Botha,H.}, Whitwell, J. L., Lowe, V., Dickson, D.
W., \& Josephs, K. A. (2014). {FDG}-{PET} in pathologically confirmed
spontaneous {4R}-tauopathy variants. \emph{Journal of Neurology},
\emph{261}(4), 710--716. \url{https://doi.org/10.1007/s00415-014-7256-4}

\leavevmode\hypertarget{ref-botha_nonverbal_2014}{}%
\textbf{Botha,H.}, Duffy, J. R., Strand, E. A., Machulda, M. M.,
Whitwell, J. L., \& Josephs, K. A. (2014). Nonverbal oral apraxia in
primary progressive aphasia and apraxia of speech. \emph{Neurology},
\emph{82}(19), 1729--1735.
\url{https://doi.org/10.1212/WNL.0000000000000412}

\leavevmode\hypertarget{ref-martinez-thompson_clinical_2014}{}%
Martinez-Thompson, J. M., \textbf{Botha,H.}, \& Katz, B. S. (2014).
Clinical {Reasoning}: {A} woman with subacute progressive confusion and
gait instability. \emph{Neurology}, \emph{83}(5), e62--e67.
\url{https://doi.org/10.1212/WNL.0000000000000648}

\leavevmode\hypertarget{ref-botha_teaching_2014}{}%
\textbf{Botha,H.}, Moore, S. A., \& Rabinstein, A. A. (2014). Teaching
{NeuroImages}: {Massive} cerebral edema after {CT} myelography: {An}
optical illusion. \emph{Neurology}, \emph{83}(18), e170--e170.
\url{https://doi.org/10.1212/WNL.0000000000000946}

\leavevmode\hypertarget{ref-botha_classification_2015}{}%
\textbf{Botha,H.}, Duffy, J. R., Whitwell, J. L., Strand, E. A.,
Machulda, M. M., Schwarz, C. G., Reid, R. I., Spychalla, A. J., Senjem,
M. L., Jones, D. T., Lowe, V., Jack, C. R., \& Josephs, K. A. (2015).
Classification and clinicoradiologic features of primary progressive
aphasia ({PPA}) and apraxia of speech. \emph{Cortex}, \emph{69},
220--236. \url{https://doi.org/10.1016/j.cortex.2015.05.013}

\leavevmode\hypertarget{ref-botha_young_2016}{}%
\textbf{Botha,H.}, Boeve, B. F., Jones, L. K., Parisi, J. E., \& Klaas,
J. P. (2016). A {Young} {Man} {With} {Progressive} {Language}
{Difficulty} and {Early}-{Onset} {Dementia}. \emph{JAMA Neurology},
\emph{73}(5), 595--599.
\url{https://doi.org/10.1001/jamaneurol.2016.0246}

\leavevmode\hypertarget{ref-yost_facial_2017}{}%
Yost, M. D., Chou, C. Z., \textbf{Botha,H.}, Block, M. S., \& Liewluck,
T. (2017). Facial diplegia after pembrolizumab treatment. \emph{Muscle
\& Nerve}, \emph{56}(3), E20--E21.
\url{https://doi.org/10.1002/mus.25663}

\leavevmode\hypertarget{ref-cohen_neurodebian_2017}{}%
Cohen, A., Kenney-Jung, D., \textbf{Botha,H.}, \& Tillema, J.-M. (2017).
{NeuroDebian} {Virtual} {Machine} {Deployment} {Facilitates}
{Trainee}-{Driven} {Bedside} {Neuroimaging} {Research}. \emph{Journal of
Child Neurology}, \emph{32}(1), 29--34.
\url{https://doi.org/10.1177/0883073816668113}

\leavevmode\hypertarget{ref-botha_novel_2017}{}%
\textbf{Botha,H.}, Finch, N. A., Gavrilova, R. H., Machulda, M. M.,
Fields, J. A., Lowe, V. J., Petersen, R. C., Jack, C. R., Dheel, C. M.,
Gearhart, D. J., Knopman, D. S., Rademakers, R., \& Boeve, B. F. (2017).
Novel {GRN} mutation presenting as an aphasic dementia and evolving into
corticobasal syndrome. \emph{Neurology Genetics}, \emph{3}(6), e201.
\url{https://doi.org/10.1212/NXG.0000000000000201}

\leavevmode\hypertarget{ref-jones_tau_2017}{}%
Jones, D. T., Graff-Radford, J., Lowe, V. J., Wiste, H. J., Gunter, J.
L., Senjem, M. L., \textbf{Botha,H.}, Kantarci, K., Boeve, B. F.,
Knopman, D. S., Petersen, R. C., \& Jack, C. R. (2017). Tau, amyloid,
and cascading network failure across the {Alzheimer}'s disease spectrum.
\emph{Cortex}, \emph{97}, 143--159.
\url{https://doi.org/10.1016/j.cortex.2017.09.018}

\leavevmode\hypertarget{ref-josephs_18fav-1451_2018}{}%
Josephs, K. A., Martin, P. R., \textbf{Botha,H.}, Schwarz, C. G., Duffy,
J. R., Clark, H. M., Machulda, M. M., Graff-Radford, J., Weigand, S. D.,
Senjem, M. L., Utianski, R. L., Drubach, D. A., Boeve, B. F., Jones, D.
T., Knopman, D. S., Petersen, R. C., Jack, C. R., Lowe, V. J., \&
Whitwell, J. L. (2018). {[}{18F}{]}{AV}-1451 tau-{PET} and primary
progressive aphasia. \emph{Annals of Neurology}, \emph{83}(3), 599--611.
\url{https://doi.org/10.1002/ana.25183}

\leavevmode\hypertarget{ref-utianski_tau_2018}{}%
Utianski, R. L., Whitwell, J. L., Schwarz, C. G., Duffy, J. R.,
\textbf{Botha,H.}, Clark, H. M., Machulda, M. M., Senjem, M. L.,
Knopman, D. S., Petersen, R. C., Jack, C. R., Lowe, V. J., \& Josephs,
K. A. (2018). Tau uptake in agrammatic primary progressive aphasia with
and without apraxia of speech. \emph{European Journal of Neurology},
\emph{25}(11), 1352--1357. \url{https://doi.org/10.1111/ene.13733}

\leavevmode\hypertarget{ref-botha_disrupted_2018}{}%
\textbf{Botha,H.}, Utianski, R. L., Whitwell, J. L., Duffy, J. R.,
Clark, H. M., Strand, E. A., Machulda, M. M., Tosakulwong, N., Knopman,
D. S., Petersen, R. C., Jack, C. R., Josephs, K. A., \& Jones, D. T.
(2018). Disrupted functional connectivity in primary progressive apraxia
of speech. \emph{NeuroImage: Clinical}, \emph{18}, 617--629.
\url{https://doi.org/10.1016/j.nicl.2018.02.036}

\leavevmode\hypertarget{ref-townley_18f-fdg_2018}{}%
Townley, R. A., \textbf{Botha,H.}, Graff-Radford, J., Boeve, B. F.,
Petersen, R. C., Senjem, M. L., Knopman, D. S., Lowe, V., Jack, C. R.,
\& Jones, D. T. (2018). {18F}-{FDG} {PET}-{CT} pattern in idiopathic
normal pressure hydrocephalus. \emph{NeuroImage: Clinical}, \emph{18},
897--902. \url{https://doi.org/10.1016/j.nicl.2018.02.031}

\leavevmode\hypertarget{ref-botha_tau-negative_2018}{}%
\textbf{Botha,H.}, Mantyh, W. G., Graff-Radford, J., Machulda, M. M.,
Przybelski, S. A., Wiste, H. J., Senjem, M. L., Parisi, J. E., Petersen,
R. C., Murray, M. E., Boeve, B. F., Lowe, V. J., Knopman, D. S., Jack,
C. R., \& Jones, D. T. (2018). Tau-negative amnestic dementia
masquerading as {Alzheimer} disease dementia. \emph{Neurology},
\emph{90}(11), e940--e946.
\url{https://doi.org/10.1212/WNL.0000000000005124}

\leavevmode\hypertarget{ref-tetzloff_clinical_2018}{}%
Tetzloff, K. A., Duffy, J. R., Strand, E. A., Machulda, M. M., Boland,
S. M., Utianski, R. L., \textbf{Botha,H.}, Senjem, M. L., Schwarz, C.
G., Josephs, K. A., \& Whitwell, J. L. (2018). Clinical and imaging
progression over 10 years in a patient with primary progressive apraxia
of speech and autopsy-confirmed corticobasal degeneration.
\emph{Neurocase}, \emph{24}(2), 111--120.
\url{https://doi.org/10.1080/13554794.2018.1477963}

\leavevmode\hypertarget{ref-botha_fdg-pet_2018}{}%
\textbf{Botha,H.}, Mantyh, W. G., Murray, M. E., Knopman, D. S.,
Przybelski, S. A., Wiste, H. J., Graff-Radford, J., Josephs, K. A.,
Schwarz, C. G., Kremers, W. K., Boeve, B. F., Petersen, R. C., Machulda,
M. M., Parisi, J. E., Dickson, D. W., Lowe, V., Jack, C. R., \& Jones,
D. T. (2018). {FDG}-{PET} in tau-negative amnestic dementia resembles
that of autopsy-proven hippocampal sclerosis. \emph{Brain},
\emph{141}(4), 1201--1217. \url{https://doi.org/10.1093/brain/awy049}

\leavevmode\hypertarget{ref-botha_non-right_2018}{}%
\textbf{Botha,H.}, Duffy, J. R., Whitwell, J. L., Strand, E. A.,
Machulda, M. M., Spychalla, A. J., Tosakulwong, N., Senjem, M. L.,
Knopman, D. S., Petersen, R. C., Jack, C. R., Lowe, V. J., \& Josephs,
K. A. (2018). Non-right handed primary progressive apraxia of speech.
\emph{Journal of the Neurological Sciences}, \emph{390}, 246--254.
\url{https://doi.org/10.1016/j.jns.2018.05.007}

\leavevmode\hypertarget{ref-utianski_prosodic_2018}{}%
Utianski, R. L., Duffy, J. R., Clark, H. M., Strand, E. A.,
\textbf{Botha,H.}, Schwarz, C. G., Machulda, M. M., Senjem, M. L.,
Spychalla, A. J., Jack, C. R., Petersen, R. C., Lowe, V. J., Whitwell,
J. L., \& Josephs, K. A. (2018). Prosodic and phonetic subtypes of
primary progressive apraxia of speech. \emph{Brain and Language},
\emph{184}, 54--65. \url{https://doi.org/10.1016/j.bandl.2018.06.004}

\leavevmode\hypertarget{ref-jones_amyloid-_2018}{}%
Jones, D. T., Townley, R. A., Graff-Radford, J., \textbf{Botha,H.},
Knopman, D. S., Petersen, R. C., Jack, C. R., Lowe, V. J., \& Boeve, B.
F. (2018). Amyloid- and tau-{PET} imaging in a familial prion kindred.
\emph{Neurology Genetics}, \emph{4}(6), e290.
\url{https://doi.org/10.1212/NXG.0000000000000290}

\leavevmode\hypertarget{ref-utianski_rapid_2018}{}%
Utianski, R. L., \textbf{Botha,H.}, Duffy, J. R., Clark, H. M., Martin,
P. R., Butts, A. M., Machulda, M. M., Whitwell, J. L., \& Josephs, K. A.
(2018). Rapid rate on quasi-speech tasks in the semantic variant of
primary progressive aphasia: {A} non-motor phenomenon? \emph{The Journal
of the Acoustical Society of America}, \emph{144}(6), 3364--3370.
\url{https://doi.org/10.1121/1.5082210}

\leavevmode\hypertarget{ref-ali_utility_2019}{}%
Ali, F., \textbf{Botha,H.}, Whitwell, J. L., \& Josephs, K. A. (2019).
Utility of the {Movement} {Disorders} {Society} {Criteria} for
{Progressive} {Supranuclear} {Palsy} in {Clinical} {Practice}.
\emph{Movement Disorders Clinical Practice}, \emph{6}(6), 436--439.
\url{https://doi.org/10.1002/mdc3.12807}

\leavevmode\hypertarget{ref-ali_sensitivity_2019}{}%
Ali, F., Martin, P. R., \textbf{Botha,H.}, Ahlskog, J. E., Bower, J. H.,
Masumoto, J. Y., Maraganore, D., Hassan, A., Eggers, S., Boeve, B. F.,
Knopman, D. S., Drubach, D., Petersen, R. C., Dunkley, E. D., Gerpen, J.
van, Uitti, R., Whitwell, J. L., Dickson, D. W., \& Josephs, K. A.
(2019). Sensitivity and {Specificity} of {Diagnostic} {Criteria} for
{Progressive} {Supranuclear} {Palsy}. \emph{Movement Disorders},
\emph{34}(8), 1144--1153. \url{https://doi.org/10.1002/mds.27619}

\leavevmode\hypertarget{ref-whitwell_evaluation_2019}{}%
Whitwell, J. L., Stevens, C. A., Duffy, J. R., Clark, H. M., Machulda,
M. M., Strand, E. A., Martin, P. R., Utianski, R. L., \textbf{Botha,H.},
Spychalla, A. J., Senjem, M. L., Schwarz, C. G., Jack, C. R., Ali, F.,
Hassan, A., \& Josephs, K. A. (2019). An {Evaluation} of the
{Progressive} {Supranuclear} {Palsy} {Speech}/{Language} {Variant}.
\emph{Movement Disorders Clinical Practice}, \emph{6}(6), 452--461.
\url{https://doi.org/10.1002/mdc3.12796}

\leavevmode\hypertarget{ref-whitwell_mri_2019}{}%
Whitwell, J. L., Tosakulwong, N., Schwarz, C. G., \textbf{Botha,H.},
Senjem, M. L., Spychalla, A. J., Ahlskog, J. E., Knopman, D. S.,
Petersen, R. C., Jack, C. R., Lowe, V. J., \& Josephs, K. A. (2019).
{MRI} {Outperforms} {[}{18F}{]}{AV}-1451 {PET} as a {Longitudinal}
{Biomarker} in {Progressive} {Supranuclear} {Palsy}. \emph{Movement
Disorders}, \emph{34}(1), 105--113.
\url{https://doi.org/10.1002/mds.27546}

\leavevmode\hypertarget{ref-graff-radford_cerebral_2019}{}%
Graff-Radford, J., \textbf{Botha,H.}, Rabinstein, A. A., Gunter, J. L.,
Przybelski, S. A., Lesnick, T., Huston, J., Flemming, K. D., Preboske,
G. M., Senjem, M. L., Brown, R. D., Mielke, M. M., Roberts, R. O., Lowe,
V. J., Knopman, D. S., Petersen, R. C., Kremers, W., Vemuri, P., Jack,
C. R., \& Kantarci, K. (2019). Cerebral microbleeds: {Prevalence} and
relationship to amyloid burden. \emph{Neurology}, \emph{92}(3),
e253--e262. \url{https://doi.org/10.1212/WNL.0000000000006780}

\leavevmode\hypertarget{ref-botha_primary_2019}{}%
\textbf{Botha,H.}, \& Josephs, K. A. (2019). Primary {Progressive}
{Aphasias} and {Apraxia} of {Speech}: \emph{CONTINUUM: Lifelong Learning
in Neurology}, \emph{25}(1), 101--127.
\url{https://doi.org/10.1212/CON.0000000000000699}

\leavevmode\hypertarget{ref-whitwell_influence_2019}{}%
Whitwell, J. L., Martin, P. R., Duffy, J. R., Clark, H. M., Machulda, M.
M., Schwarz, C. G., Weigand, S. D., Sintini, I., Senjem, M. L.,
Ertekin-Taner, N., \textbf{Botha,H.}, Utianski, R. L., Graff-Radford,
J., Jones, D. T., Boeve, B. F., Knopman, D. S., Petersen, R. C., Jack,
C. R., Lowe, V. J., \& Josephs, K. A. (2019). The influence of
\(\beta\)-amyloid on {[} \(^{\textrm{18}}\) {F}{]}{AV}-1451 in semantic
variant of primary progressive aphasia. \emph{Neurology}, \emph{92}(7),
e710--e722. \url{https://doi.org/10.1212/WNL.0000000000006913}

\leavevmode\hypertarget{ref-tetzloff_progressive_2019}{}%
Tetzloff, K. A., Duffy, J. R., Clark, H. M., Utianski, R. L., Strand, E.
A., Machulda, M. M., \textbf{Botha,H.}, Martin, P. R., Schwarz, C. G.,
Senjem, M. L., Reid, R. I., Gunter, J. L., Spychalla, A. J., Knopman, D.
S., Petersen, R. C., Jack, C. R., Lowe, V. J., Josephs, K. A., \&
Whitwell, J. L. (2019). Progressive agrammatic aphasia without apraxia
of speech as a distinct syndrome. \emph{Brain}, \emph{142}(8),
2466--2482. \url{https://doi.org/10.1093/brain/awz157}

\leavevmode\hypertarget{ref-townley_comparison_2019}{}%
Townley, R. A., Syrjanen, J. A., \textbf{Botha,H.}, Kremers, W. K.,
Aakre, J. A., Fields, J. A., Machulda, M. M., Graff-Radford, J., Savica,
R., Jones, D. T., Knopman, D. S., Petersen, R. C., \& Boeve, B. F.
(2019). Comparison of the {Short} {Test} of {Mental} {Status} and the
{Montreal} {Cognitive} {Assessment} {Across} the {Cognitive} {Spectrum}.
\emph{Mayo Clinic Proceedings}, \emph{94}(8), 1516--1523.
\url{https://doi.org/10.1016/j.mayocp.2019.01.043}

\leavevmode\hypertarget{ref-sintini_multimodal_2019}{}%
Sintini, I., Schwarz, C. G., Senjem, M. L., Reid, R. I.,
\textbf{Botha,H.}, Ali, F., Ahlskog, J. E., Jack, C. R., Lowe, V. J.,
Josephs, K. A., \& Whitwell, J. L. (2019). Multimodal neuroimaging
relationships in progressive supranuclear palsy. \emph{Parkinsonism \&
Related Disorders}, \emph{66}, 56--61.
\url{https://doi.org/10.1016/j.parkreldis.2019.07.001}

\leavevmode\hypertarget{ref-jack_bivariate_2019}{}%
Jack, C. R., Wiste, H. J., \textbf{Botha,H.}, Weigand, S. D., Therneau,
T. M., Knopman, D. S., Graff-Radford, J., Jones, D. T., Ferman, T. J.,
Boeve, B. F., Kantarci, K., Lowe, V. J., Vemuri, P., Mielke, M. M.,
Fields, J. A., Machulda, M. M., Schwarz, C. G., Senjem, M. L., Gunter,
J. L., \& Petersen, R. C. (2019). The bivariate distribution of
amyloid-\(\beta\) and tau: Relationship with established neurocognitive
clinical syndromes. \emph{Brain}, \emph{142}(10), 3230--3242.
\url{https://doi.org/10.1093/brain/awz268}

\leavevmode\hypertarget{ref-utianski_clinical_2019}{}%
Utianski, R. L., \textbf{Botha,H.}, Martin, P. R., Schwarz, C. G.,
Duffy, J. R., Clark, H. M., Machulda, M. M., Butts, A. M., Lowe, V. J.,
Jack, C. R., Senjem, M. L., Spychalla, A. J., Whitwell, J. L., \&
Josephs, K. A. (2019). Clinical and neuroimaging characteristics of
clinically unclassifiable primary progressive aphasia. \emph{Brain and
Language}, \emph{197}, 104676.
\url{https://doi.org/10.1016/j.bandl.2019.104676}

\leavevmode\hypertarget{ref-bejanin_antemortem_2019}{}%
Bejanin, A., Murray, M. E., Martin, P., \textbf{Botha,H.}, Tosakulwong,
N., Schwarz, C. G., Senjem, M. L., Chételat, G., Kantarci, K., Jack, C.
R., Boeve, B. F., Knopman, D. S., Petersen, R. C., Giannini, C., Parisi,
J. E., Dickson, D. W., Whitwell, J. L., \& Josephs, K. A. (2019).
Antemortem volume loss mirrors {TDP}-43 staging in older adults with
non-frontotemporal lobar degeneration. \emph{Brain}, \emph{142}(11),
3621--3635. \url{https://doi.org/10.1093/brain/awz277}

\leavevmode\hypertarget{ref-townley_progressive_2020}{}%
Townley, R. A., Graff-Radford, J., Mantyh, W. G., \textbf{Botha,H.},
Polsinelli, A. J., Przybelski, S. A., Machulda, M. M., Makhlouf, A. T.,
Senjem, M. L., Murray, M. E., Reichard, R. R., Savica, R., Boeve, B. F.,
Drubach, D. A., Josephs, K. A., Knopman, D. S., Lowe, V. J., Jack, C.
R., Petersen, R. C., \& Jones, D. T. (2020). Progressive dysexecutive
syndrome due to {Alzheimer}{'}s disease: A description of 55 cases and
comparison to other phenotypes. \emph{Brain Communications},
\emph{2}(1), fcaa068. \url{https://doi.org/10.1093/braincomms/fcaa068}

\leavevmode\hypertarget{ref-whitwell_brain_2020}{}%
Whitwell, J. L., Tosakulwong, N., \textbf{Botha,H.}, Ali, F., Clark, H.
M., Duffy, J. R., Utianski, R. L., Stevens, C. A., Weigand, S. D.,
Schwarz, C. G., Senjem, M. L., Jack, C. R., Lowe, V. J., Ahlskog, J. E.,
Dickson, D. W., \& Josephs, K. A. (2020). Brain volume and flortaucipir
analysis of progressive supranuclear palsy clinical variants.
\emph{NeuroImage: Clinical}, \emph{25}, 102152.
\url{https://doi.org/10.1016/j.nicl.2019.102152}

\leavevmode\hypertarget{ref-whitwell_longitudinal_2020}{}%
Whitwell, J. L., Tosakulwong, N., Weigand, S. D., Graff-Radford, J.,
Duffy, J. R., Clark, H. M., Machulda, M. M., \textbf{Botha,H.},
Utianski, R. L., Schwarz, C. G., Senjem, M. L., Strand, E. A.,
Ertekin-Taner, N., Jack Jr, C. R., Lowe, V. J., \& Josephs, K. A.
(2020). Longitudinal {Amyloid}-\(\beta\) {PET} in {Atypical}
{Alzheimer}{'}s {Disease} and {Frontotemporal} {Lobar} {Degeneration}.
\emph{Journal of Alzheimer's Disease}, \emph{74}(1), 377--389.
\url{https://doi.org/10.3233/JAD-190699}

\leavevmode\hypertarget{ref-clark_heather_m_western_2020}{}%
Clark Heather M., Utianski Rene L., Duffy Joseph R., Strand Edythe A.,
Botha Hugo, Josephs Keith A., \& Whitwell Jennifer L. (2020). Western
{Aphasia} {Battery}{{}}{Revised} {Profiles} in {Primary} {Progressive}
{Aphasia} and {Primary} {Progressive} {Apraxia} of {Speech}.
\emph{American Journal of Speech-Language Pathology}, \emph{29}(1S),
498--510. \url{https://doi.org/10.1044/2019_AJSLP-CAC48-18-0217}

\leavevmode\hypertarget{ref-utianski_longitudinal_2020}{}%
Utianski, R. L., Martin, P. R., \textbf{Botha,H.}, Schwarz, C. G.,
Duffy, J. R., Petersen, R. C., Knopman, D. S., Clark, H. M., Butts, A.
M., Machulda, M. M., Jack, C. R., Lowe, V. J., Whitwell, J. L., \&
Josephs, K. A. (2020). Longitudinal flortaucipir ({[}{18F}{]}{AV}-1451)
{PET} imaging in primary progressive apraxia of speech. \emph{Cortex},
\emph{124}, 33--43. \url{https://doi.org/10.1016/j.cortex.2019.11.002}

\leavevmode\hypertarget{ref-boes_dementia_2020}{}%
Boes, S., \textbf{Botha,H.}, Machulda, M., Lowe, V., Graff-Radford, J.,
Whitwell, J. L., Utianski, R. L., Duffy, J. R., \& Josephs, K. A.
(2020). Dementia with {Lewy} bodies presenting as {Logopenic} variant
primary progressive {Aphasia}. \emph{Neurocase}, \emph{0}(0), 1--5.
\url{https://doi.org/10.1080/13554794.2020.1795204}

\leavevmode\hypertarget{ref-buciuc_utility_2020}{}%
Buciuc, M., \textbf{Botha,H.}, Murray, M. E., Schwarz, C. G., Senjem, M.
L., Jones, D. T., Knopman, D. S., Boeve, B. F., Petersen, R. C., Jack,
C. R., Petrucelli, L., Parisi, J. E., Dickson, D. W., Lowe, V.,
Whitwell, J. L., \& Josephs, K. A. (2020). Utility of {FDG}-{PET} in
diagnosis of {Alzheimer}-related {TDP}-43 proteinopathy.
\emph{Neurology}, \emph{95}(1), e23--e34.
\url{https://doi.org/10.1212/WNL.0000000000009722}

\leavevmode\hypertarget{ref-clark_dysphagia_2020}{}%
Clark, H. M., Stierwalt, J. A. G., Tosakulwong, N., \textbf{Botha,H.},
Ali, F., Whitwell, J. L., \& Josephs, K. A. (2020). Dysphagia in
{Progressive} {Supranuclear} {Palsy}. \emph{Dysphagia}, \emph{35}(4),
667--676. \url{https://doi.org/10.1007/s00455-019-10073-2}

\leavevmode\hypertarget{ref-ghirelli_sensitivity-specificity_2020}{}%
Ghirelli, A., Tosakulwong, N., Weigand, S. D., Clark, H. M., Ali, F.,
\textbf{Botha,H.}, Duffy, J. R., Utianski, R. L., Buciuc, M., Murray, M.
E., Labuzan, S. A., Spychalla, A. J., Pham, N. T. T., Schwarz, C. G.,
Senjem, M. L., Machulda, M. M., Baker, M., Rademakers, R., Filippi, M.,
\ldots{} Whitwell, J. L. (2020). Sensitivity-{Specificity} of {Tau} and
{Amyloid} \(\beta\) {Positron} {Emission} {Tomography} in
{Frontotemporal} {Lobar} {Degeneration}. \emph{Annals of Neurology}.
\url{https://doi.org/10.1002/ana.25893}

\leavevmode\hypertarget{ref-utianski_longitudinal_2020-1}{}%
Utianski, R. L., \textbf{Botha,H.}, Whitwell, J. L., Martin, P. R.,
Schwarz, C. G., Duffy, J. R., Clark, H. M., Spychalla, A. J., Senjem, M.
L., Petersen, R. C., Knopman, D. S., Jack, C. R., Lowe, V. J., \&
Josephs, K. A. (2020). Longitudinal flortaucipir ({[}{18F}{]}{AV}-1451)
{PET} uptake in semantic dementia. \emph{Neurobiology of Aging},
\emph{92}, 135--140.
\url{https://doi.org/10.1016/j.neurobiolaging.2020.04.010}

\leavevmode\hypertarget{ref-whitwell_survival_2020}{}%
Whitwell, J. L., Martin, P., Duffy, J. R., Clark, H. M., Utianski, R.
L., \textbf{Botha,H.}, Machulda, M. M., Strand, E. A., \& Josephs, K. A.
(2020). Survival analysis in primary progressive apraxia of speech and
agrammatic aphasia. \emph{Neurology: Clinical Practice},
10.1212/CPJ.0000000000000919.
\url{https://doi.org/10.1212/CPJ.0000000000000919}

\leavevmode\hypertarget{ref-seckin_ioflupane_2020}{}%
Seckin, Z. I., Whitwell, J. L., Utianski, R. L., \textbf{Botha,H.}, Ali,
F., Duffy, J. R., Clark, H. M., Machulda, M. M., Jordan, L. G., Min,
H.-K., Lowe, V. J., \& Josephs, K. A. (2020). Ioflupane {123I} ({DAT}
scan) {SPECT} identifies dopamine receptor dysfunction early in the
disease course in progressive apraxia of speech. \emph{Journal of
Neurology}, \emph{267}(9), 2603--2611.
\url{https://doi.org/10.1007/s00415-020-09883-4}

\leavevmode\hypertarget{ref-cogswell_associations_2020}{}%
Cogswell, P. M., Wiste, H. J., Senjem, M. L., Gunter, J. L., Weigand, S.
D., Schwarz, C. G., Arani, A., Therneau, T. M., Lowe, V. J., Knopman, D.
S., \textbf{Botha,H.}, Graff-Radford, J., Jones, D. T., Kantarci, K.,
Vemuri, P., Boeve, B. F., Mielke, M. M., Petersen, R. C., \& Jack, C. R.
(2020). Associations of {Quantitative} {Susceptibility} {Mapping} with
{Alzheimer}'s {Disease} {Clinical} and {Imaging} {Markers}.
\emph{NeuroImage}, 117433.
\url{https://doi.org/10.1016/j.neuroimage.2020.117433}

\leavevmode\hypertarget{ref-jack_predicting_2020}{}%
Jack, C. R., Wiste, H. J., Weigand, S. D., Therneau, T. M., Lowe, V. J.,
Knopman, D. S., \textbf{Botha,H.}, Graff-Radford, J., Jones, D. T.,
Ferman, T. J., Boeve, B. F., Kantarci, K., Vemuri, P., Mielke, M. M.,
Whitwell, J., Josephs, K., Schwarz, C. G., Senjem, M. L., Gunter, J. L.,
\& Petersen, R. C. (2020). Predicting future rates of tau accumulation
on {PET}. \emph{Brain : A Journal of Neurology}, \emph{143}(10),
3136--3150. \url{https://doi.org/10.1093/brain/awaa248}

\leavevmode\hypertarget{ref-seckin_evolution_2020}{}%
Seckin, Z. I., Duffy, J. R., Strand, E. A., Clark, H. M., Utianski, R.
L., Machulda, M. M., \textbf{Botha,H.}, Ali, F., Thu Pham, N. T., Lowe,
V. J., Whitwell, J. L., \& Josephs, K. A. (2020). The evolution of
parkinsonism in primary progressive apraxia of speech: {A} 6-year
longitudinal study. \emph{Parkinsonism \& Related Disorders}, \emph{81},
34--40. \url{https://doi.org/10.1016/j.parkreldis.2020.09.039}

\leavevmode\hypertarget{ref-sintini_tau_2020}{}%
Sintini, I., Graff-Radford, J., Jones, D. T., \textbf{Botha,H.}, Martin,
P. R., Machulda, M. M., Schwarz, C. G., Senjem, M. L., Gunter, J. L.,
Jack, C. R., Lowe, V. J., Josephs, K. A., \& Whitwell, J. L. (2020). Tau
and {Amyloid} {Relationships} with {Resting}-state {Functional}
{Connectivity} in {Atypical} {Alzheimer}'s {Disease}. \emph{Cerebral
Cortex (New York, N.Y. : 1991)}.
\url{https://doi.org/10.1093/cercor/bhaa319}

\leavevmode\hypertarget{ref-utianski_communication_2020}{}%
Utianski, R. L., Clark, H. M., Duffy, J. R., \textbf{Botha,H.},
Whitwell, J. L., \& Josephs, K. A. (2020). Communication {Limitations}
in {Patients} {With} {Progressive} {Apraxia} of {Speech} and {Aphasia}.
\emph{American Journal of Speech-Language Pathology}, \emph{29}(4),
1976--1986. \url{https://doi.org/10.1044/2020_AJSLP-20-00012}

\leavevmode\hypertarget{ref-utianski_longitudinal_2021}{}%
Utianski, R. L., Martin, P. R., Hanley, H., Duffy, J. R.,
\textbf{Botha,H.}, Clark, H. M., Whitwell, J. L., \& Josephs, K. A.
(2021). A {Longitudinal} {Evaluation} of {Speech} {Rate} in {Primary}
{Progressive} {Apraxia} of {Speech}. \emph{Journal of Speech, Language,
and Hearing Research : JSLHR}, 1--13.
\url{https://doi.org/10.1044/2020_JSLHR-20-00253}

\endgroup

\end{document}
